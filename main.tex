\documentclass{article}
\usepackage[utf8]{inputenc}

\title{Secure Programming Summary}
\author{j.s.burley }
\date{November 2018}

\begin{document}

\maketitle

\tableofcontents
\newpage
\section{Introduction}
    \subsection{What is security?}
    The state of being secure
    \\CIA: \textbf{only} authorised actors can:
    \begin{itemize}
        \item Learn secrets (\textbf{Confidentiality})
        \begin{itemize}
            \item \textbf{Data Confidentiality}: make sure that private information is not available to unauthorised actors
            \item \textbf{Privacy}: Individuals have control over what information is collected and stored, and how that information is shared
        \end{itemize}
        \item Modify messages/data (\textbf{Integrity})
        \begin{itemize}
            \item \textbf{Data Integrity}: Assures that information is modified only in a way that is known
            \item \textbf{System Integrity}: Ensures that a system performs as designed, without unauthorised modification
        \end{itemize}
        \item Access messages/data (\textbf{Availability})
        \begin{itemize}
            \item \textbf{Availability}: ensures that service is provided to an acceptable level, without denying access to authorised users
        \end{itemize}
    \end{itemize}
    Security is best \textbf{by design} - not as afterthought
    \subsection{Identifying requirements for a secure protocol}
    \begin{itemize}
        \item Always work towards CIA protocols described above
        \item We can do this by using the following three techniques
        \begin{itemize}
            \item Cryptographic algorithms to hide, sign and provide guarantees about messages
            \item Use cryptography in protocols, systems in order to secure them
            \item Program with security in mind in order to avoid exploits that may manipulate software
        \end{itemize}
        \item An additional requirement is \textbf{Non-repudiation}: One party of a transaction cannot deny having received a transaction, nor can the other party deny having sent the transaction.
    \end{itemize}
    Access Control
    \begin{itemize}
        \item \textbf{Identification}: A user states their identity (i.e. a username)
        \item \textbf{Authentication}: The system verifies this identity (i.e. through a shared secret like a password)
        \item \textbf{Authorisation}: This user is authorised to perform a particular task, or access a particular file
        \end{itemize}
\section{Cryptology}
Cryptography: The mathematics of secret communication
\\Cryptology: Includes study of breaking cryptographic protocols (known as cryptanalysis)
\\The challenge of cryptology is to map mathematics of cryptosystems onto computers with the help of engineers
\subsection{Basic Concepts}
\textbf{Kerckhoff's Principle}: A cryptosystem should be secure \textit{even if} everything about the system, \textit{except the key} is public knowledge.
\\There is no "Security by obscurity": all crypto algorithms are assumed to be known. Security is based on:
\begin{itemize}
    \item Secrecy of the key
    \item Hard to infer the plaintext from the cipher text
\end{itemize}
\textbf{Cryptanalysis} is inferring the plaintext from ciphertext \textit{without} knowing the key.
\subsection{Examples of ciphers}
\subsubsection{Caesar Cipher}
\begin{itemize}
    \item Monoalphabetic substitution cipher
    \item Key is a constant shift (i.e. 5 letters)
    \item We encrypt by shifting each message letter by $key$ letters (AAA -> FFF)
    \item ROT13 is least secure variant of this: shift by 13 letters. ROT13 is its own inverse
\end{itemize}
It is easy to break monoalphabetic substitution ciphers using techniques like frequency analysis of the most common letters in a given language. We can then perform frequency analysis on the message to determine if there is a letter that matches the frequency of one in the message language.
\subsubsection{Vigenére Cipher}
\begin{itemize}
    \item Polyalphabetic substitution cipher
    \item We use a table called a \textit{tabula recta}
    \item Has multiple keys
    \item Repeat the keys for the length of the message, use the table to determine which letter to substitute
\end{itemize}
We can break polyalphabetic ciphers by breaking the message into components and grouping ciphertexts into parts corresponding to each key character. We can then use frequency analysis on each of these message components.
\subsubsection{One Time Pad}
\begin{itemize}
    \item Takes concept of Vigenére to the extreme
    \item Key size = plaintext size (or larger) in order to avoid repeated patterns
    \item Only known algorithm with perfect secrecy
    \item Key distribution is a problem (have to keep key secret, prevent interception)
    \item If random key is never re-used as a whole or in part, and kept completely secret, impossible to use cryptanalysis to break it.
\end{itemize}
\subsubsection{Block ciphers}
With the above methods, we treat messages as a one-dimensional stream, and we either substitute or shift letters to encrypt. With \textbf{Block Ciphers}, we include transpositions.
\subsubsection{Playfair Cipher}
\begin{itemize}
    \item simple block cipher from 1854 (unsafe nowadays)
    \item Key is a 5x5 matrix with the keyphrase (we replace I with J so the alphabet fits)
    \item encrypt messages in digrams (two letter pairs)
    \item pad the message if a digram is incomplete, or if a digram consists of two of the same letter (X is a common padding letter)
\end{itemize}
Rules of playfair:
\begin{enumerate}
    \item If two letters are in the same row or column, replace by shifting the letter to the right (row) or down (column). Use wrap around if there is no letter directly to the right of a chosen letter.
    \item Otherwise, picture a rectangle formed by the two letters. Replace the letters with those at the two "unoccupied" corners of that rectangle. The order for this is important: first letter of ciphertext pair must come from the same \textbf{row} as the first letter of the plaintext pair.
\end{enumerate}


\subsection{Symmetric Encryption}
\subsubsection{Symmetric Ciphers}
\begin{itemize}
    \item relatively fast
    \item One key to encrypt and decrypt
    \item block vs stream variants
    \item has a major weakness: \textbf{key distribution}
\end{itemize}
\subsubsection{Modern Symmetric Ciphers}
Typically injective (one-to-one) to allow for decryption
\\Vulnerable to statistical attacks, as small blocks can only take limited transformations
\\large blocks are impractical
\\Key size can grow quite large: 4 bits x 16 rows
\\in general, n x $2^n$, so a 64-bit block would require a key of 64 x $2^64 = 10^21$ bits (125 Exabytes, 125,000 petabytes)
\begin{itemize}
    \item DES, 3DES, AES (AES most dominant, DES broken)
    \item based on substitution and transposition
    \item too complex to do by hand
    \item block ciphers (DES, 3DES, AES,...)
    \item also stream ciphers, the most well known being RC4
\end{itemize}
\subsubsection{Block vs Stream}
\begin{itemize}
    \item Block ciphers: block of plaintext treated as a whole and produces a ciphertext block (typically of equal length). Typical length is 64 or 128 bits. Achieved using substitution and transposition. Diffusion \& confusion
    \item Stream ciphers: encrypts a digital data stream 1 bit/byte at a time, no regard for padding or length
\end{itemize}
\subsubsection{Diffusion vs Confusion}
\begin{itemize}
    \item diffusion: each plaintext digit affects the value of many ciphertext digits with a cascade effect
    \item confusion: statistics of ciphertext and value of the encryption key are as complex as possible
\end{itemize}
\subsubsection{Transposition cipher}
Permutation over a block of plaintext.
\begin{verbatim}
    Key:        4312567
    Plaintext:  attackp
                ostpone
                duntilt
                woamxyz
\end{verbatim}
If you read the characters column-wise in the order provided by the key, you get the following message:
\begin{verbatim}
    ttna aptm tsuo aodw coix knly petz
\end{verbatim}
Putting this together, we get the nonsense string:
\begin{verbatim}
    ttnaaptmtsuoaodwcoixknlypetz
\end{verbatim}
\subsubsection{Feistel Cipher}
\begin{itemize}
    \item Goal: approximate ideal cipher and reduce statistical properties linking plaintext, ciphertext and keys
    \item Combining substitutions and permutations: each plaintext element or group of elements is uniquely replaced by a corresponding ciphertext element or group of elements, and a sequence of plaintext elements is replaced by a permutation of that sequence. No elements are added or deleted or replaced in the sequence, but the order of the elements in the sequence is changed.
\end{itemize}
\textbf{Substitution:} right part of plaintext block transformed by F(Ki) and XORed with left part
\\\textbf{Permutation:} right part swapped with left part
\\Relationship between the $i^{th}$ round and the output of the previous round given by:
$$L_i = R_{i-1}$$
$$R_i = L_{i-1}\bigoplus f(K_i, R_{i-1})$$
where $f$ is called the Feistel function.
\textbf{Properties}
\begin{itemize}
    \item \textbf{Block size:} larger blocks mean greater security but lower encrypt/decrypt speed. Block size of 64 is a reasonable tradeoff, AES uses 128
    \item \textbf{Key size:} larger key = greater security but reduced encrypt/decrypt speed. 64-bit key size is considered inadequate, 128bits common
    \item \textbf{Number of rounds:} essence of feistel cipher is that a single round is not secure enough, multiple rounds increase security. 16 rounds is typical
    \item \textbf{Round key generation algorithm:} greater complexity in this algorithm -> greater difficulty of cryptanalysis
    \item \textbf{Round function F:} greater complexity in this algorithm -> greater difficulty of cryptanalysis
\end{itemize}
\textbf{Extra (desired) properties}
\begin{itemize}
    \item \textbf{Fast software encryption/decryption:} encryption is often embedded in applications or utility functions to avoid a hardware implementation
    \item \textbf{ease of analysis:} If algorithm is easier to analyze, it is easier to find (cryptanalytic) vulnerabilities and strengthen algorithm. (i.e. DES is not easily analyzed)
\end{itemize}
\textbf{Block modes}
\\\textbf{Electronic codebook (ECB)}
\begin{itemize}
    \item each block is encoded independently using the same key
    \item used for: secure transmission of single blocks (i.e. an encryption key)
    \item \textbf{identical blocks can leak information or lead to attacks!}
\end{itemize}
\textbf{Cipher Block Chaining (CBC)}
\begin{itemize}
    \item input to the encryption algorithm is the XOR of the next 64 bits of plaintext and the preceding 64 bits of ciphertext
    \item used for: general purpose block-oriented transmission
    \item Requires initialization vector (IV) to XOR with the first plaintext block
\end{itemize}
There are others: PCBC, CFB, OFB, CTR.
\\\textbf{Choosing the wrong block mode is a major factor in implementation weaknesses}
\subsubsection{DES}
\begin{itemize}
    \item 64 bit block size
    \item key length is 56 bits, with 8 bits for parity (64 bits total)
    \item 16 round feistel structure
\end{itemize}
\subsubsection{AES}
\begin{itemize}
    \item subset of \textit{Rijndael}, developed in 1998 by two belgian cryptographers John Daemen and Vincent Rijmen
    \item most widely used symmetric cipher today
    \item 128 bit block size
    \item 128, 192 or 256 bit key size
\end{itemize}
\section{Public Key Crypto Systems}
\begin{itemize}
    \item public-key/two-key/asymmetric crypto involves the use of two keys
    \item \textbf{public key:} may be known to anyone, and can be used to encrypt messages and verify signatures
    \item \textbf{private key:} known only to the recipient, used to decrypt messages and create signatures
    \item \textbf{asymmetric because:} encryptors and verifiers cannot decrypt or create signatures
    \item {PKCS: Public Key Crypto Standards}
\end{itemize}
\subsection{Why do we use it?}
Addresses two key issues:
\begin{itemize}
    \item \textbf{Key distribution:} how to have secure communications in general without having to trust the channel when transmitting symmetric key
    \item \textbf{digital signatures:} how do we verify a message comes (intact) from the claimed sender?
    \item Publicly invented by Whitfield Diffie and Martin Hellman in 1976, although existence of concept was known prior
\end{itemize}
It also has some useful features
\begin{itemize}
    \item can be used to transmit message securely
    $$C = E(M,P_{pu})$$
    $$M = D(M,P_{pr})$$
    Where C is ciphertext, M is message, and E and D are decrypt with P as the public and private key
    \item can also be used for authentication using a combination of pubkey/privkey encryption/decryption
    $$C' = E(M,P^{A}_{pu})$$
    $$C = E(M||C',P^{B}_{pu})$$
\end{itemize}
\subsection{Characteristics}
pubkey algorithms rely on two keys where:
\begin{itemize}
    \item its computationally infeasible to find decryption key knowing only algorithm and encryption key
    \item computationally easy to en/de-crypt messages when relevant key is known
    \item (for some algorithms) either key can be used to encrypt, with private required to decrypt
\end{itemize}
\subsection{RSA}
by Rivest, Shamir and Adleman of MIT in 1977
\\best known, most widely used pubkey scheme
\\based on exponentiation in a finite (Galois) field over integers modulo a prime (exponentation takes $O((log n)^3)$ operations, computationally easy)
\\uses large (1024 bits) integers
\\security through cost of factoring large numbers (factorization takes $O(e^{log n log n log n}$ operations, computationally hard)
\subsubsection{Setting up an RSA key}
Each user must generate a public/private key pair by:
\begin{itemize}
    \item Selecting two large primes at random - $p, q$
    \item compute their modulus $n=p\times q$, where $\phi(n)=(p-1)(q-1)$ \footnote{$\phi(n)$ being Euler's totient function: the number of integers $1 \leq k \leq n$ that are relatively prime to $n$}
    \item select at random the encryption key $e$ where $1 < \phi(n)$, $gcd(e,\phi(n)) = 1$
    \item solve the following equation to find decryption key $d$: $e \times d = 1 mod \phi(n)$ and $0<d<\phi(n)$
\end{itemize}
Users publish their encryption key: $PU ={e,n}$
\\Users keep private decryption key secret: $PR={d,n}$
\subsubsection{RSA usage}
To \textbf{encrypt} a message M, the sender:
\begin{itemize}
    \item Obtains the \textbf{public key} of recipient $PU$
    \item Computes: $C=M^e mod n$, where $0\leq M < n$
\end{itemize}
To decrypt the ciphertext C, the receiver:
\begin{itemize}
    \item uses their \textbf{private key} $PR={d,n}$
    \item computes: $M = C^d mod n$
\end{itemize}
\textit{note: message M must be smaller than modulus n (split message into blocks if necessary)}
\subsubsection{Why does RSA work like this?}
\textbf{Euler's totient function}
Expressed as $\phi(n)$ or $\varphi(n)$
\\Defined as the number of positive integers less than n and relatively prime to n
\\$\varphi(1) = 1$
\\Given that $n=pq$, and $p$ and $q$ are prime numbers:
$$\varphi(n)=\varphi(pq)=\varphi(p)\varphi(q)=(p-1)(q-1)$$
\textbf{Euler's theorem}
$\alpha, n$ relative prime: the only positive integer that evenly divides both is 1
$$\alpha^{\phi(n)}\equiv 1(mod n)$$
\textbf{So why does RSA work?}
Because of Euler's theorem:
$$a^{\phi(n)}mod n = 1$$ where $gcd(a,n)=1$
\\in RSA we have:
$$n=p\times q$$
$$\phi(n)=(p-1)(q-1)$$
we carefully choose $e$ and $d$ to be inverses $mod\phi(n)$
\\so, $e\times d = 1+k \times \phi(n)$ for some $k$
\\hence:
$$C^d = M^{e\times d} = M^{1+k\times \phi(n)} = M^1 \times (M^{\phi(n)})^k$$
$$= M^1 \times (1)^k = M^1 = M mod n$$
\subsubsection{Practical considerations}
Messages need to be prepared/encoded
\\Plaintext message units: all blocks of $k$ length, regarded as k-digit base n integers (i.e. we assign them numerical equivalents between $0$ and $N^k$)
\\Ciphertext message units: blocks of $\ell$ letters in our N-letter alphabet.
\subsubsection{RSA security}
There are some ways we could try and attack RSA:
\begin{itemize}
    \item brute force key search (infeasible given size of numbers)
    \item mathematical attacks (based on difficulty of computing $\phi(n)$, by factoring modulus n)
    \item timing attacks (on running of decryption)
    \item \textbf{chosen ciphertext attacks (given properties of RSA)}
\end{itemize}
\subsubsection{RSA factoring problem}
3 forms of mathematical approach
\begin{itemize}
    \item factor $n=p\times q$, hence computing $\phi(n)$ and then $d$
    \item determine $\phi(n)$ directly and compute $d$
    \item find $d$ directly
\end{itemize}
Currently, it is believed all attacks use factoring
\\Slow improvements over the years
\\As of 05/05/2017, keys of 200 decimal digits (663 bits) have been broken
\\We currently assume 1024-2048 bit RSA is secure (assuming p and q of a similiar size and matching other constraints)
\subsubsection{Chosen Ciphertext Attacks as a way to break RSA}
RSA is vulnerable to CCA
\\Assume an oracle where the attacker can choose a ciphertext and get a decrypted plaintext back
\\Choose ciphertext to exploit properties of RSA to provide info to help cryptanalysis
\\Can counter with random pad of plaintext, or use \textbf{Optimal Asymmetric Encryption Padding} (OASP)
\end{document}
